\documentclass[10pt,oneside,fleqn]{scrartcl}
\usepackage{myhandout}
\usepackage[%
  backend=biber,
  bibencoding=latin1,
  style=ieee, % ieee, nature, numeric, authoryear いろいろある
  url=false, % 余計な項目は表示しない
  isbn=false,
  doi=false,
  eprint=false,
]{biblatex}
\AtEveryBibitem{\clearfield{note}} % note項目を表示しない
\addbibresource{books.bib} 
% \addbibresource{papers.bib} % databaseを追加する場合

%%% 必要なpackageの読み込みの例
\usepackage{graphicx}
\graphicspath{{./imgs/}}
\usepackage{bxjalipsum} % ダミーの文書
\usepackage[math]{blindtext} % ダミーの文書
%%% packageの例の終わり

%%% タイトル
\author{夏目 漱石}
\date{\today}
\title{「我輩」の秘密に関する研究\\\medskip
\large 私が彼について知っている2,3の事柄}

\begin{document}

\maketitle
\section{我輩のこと}
\jalipsum[1-3]{wagahai}
\sidenote{\jalipsum{iroha}}

\jalipsum[4-6]{wagahai}
\sidenote{例えば\cite{吉田2006,竹内1963,杉浦1980,杉浦1985,田中2006}を参照}

\section{主人の不平}
\jalipsum[7-12]{wagahai}
%% 図を入れてみる
\GraphFile{sample_figs} % 紙芝居をする場合
\begin{figure}[htbp] % 普通の環境
  \sidecaption{東京都の陽性患者数の推移.
    緑は7日移動平均,橙は14日移動平均を表す.
    \label{fig:1}}
  \centering
  \myGraph[1]{} % linewidthの何倍か指定
\end{figure}
%% 図の上に sidecaption を置く場合

\section{車屋の黒猫}
\jalipsum[13-18]{wagahai}

\begin{figure*} % 目一杯拡げて2つ並べてみる
  \centering
  \myGraph{状態空間モデルによる各成分の推定}
  \myGraph{状態空間モデルによる平均の推定}
  \sidecaption{状態空間モデルの推定.
    \label{fig:2}}
\end{figure*}

\section{鼠以外の御馳走}
\jalipsum[19-24]{wagahai}
\begin{marginfigure}
  \centering
  \myGraph*{} % marginに合わせて表示
  \caption{$Q_{\mathrm{cycle}}$ の検討について.
    \label{fig:4}}
\end{marginfigure}

\section{放蕩家について}
\jalipsum[25-]{wagahai}

\begin{otherlanguage}{english}
  % babel系が若干悪さをするので英語にして回避
  \printbibliography[title=参考文献]
\end{otherlanguage}

\end{document}