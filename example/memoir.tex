% Created 2021-11-13 Sat 00:27
% Intended LaTeX compiler: pdflatex
\documentclass[10pt,oneside,fleqn]{scrartcl}
                  

\usepackage{myhandout}
\usepackage[backend=biber,bibencoding=latin1,style=ieee,url=false,isbn=false,doi=false,eprint=false]{biblatex}
\AtEveryBibitem{\clearfield{note}} % note項目を表示しない
\addbibresource{books.bib}
\usepackage{bxjalipsum} % ダミーの文書
\usepackage[math]{blindtext} % ダミーの文書
\author{夏目 漱石}
\date{\today}
\title{「我輩」の秘密に関する研究\\\medskip
\large 私が彼について知っている2,3の事柄}
\hypersetup{
 pdfauthor={夏目 漱石},
 pdftitle={「我輩」の秘密に関する研究},
 pdfkeywords={},
 pdfsubject={},
 pdfcreator={Emacs 27.2 (Org mode 9.4.6)}, 
 pdflang={Japanese}}
\begin{document}

\maketitle

\section{我輩のこと}
\label{sec:org17b614a}
\jalipsum[1-3]{wagahai} \footnote{\jalipsum{iroha}}

\jalipsum[4-6]{wagahai} \footnote{例えば\cite{吉田2006,竹内1963,杉浦1980,杉浦1985,田中2006}を参照}

\section{主人の不平}
\label{sec:orgae60009}
\jalipsum[7-12]{wagahai}

\GraphFile{sample_figs} 
\begin{figure}[htbp] % 普通の環境 図の上に sidecaption を置く場合
  \sidecaption{東京都の陽性患者数の推移.
    緑は7日移動平均,橙は14日移動平均を表す.
    \label{fig:1}}
  \centering
  \myGraph[1]{} % linewidthの何倍か指定
\end{figure}

\section{車屋の黒猫}
\label{sec:orgafbc8b3}
\jalipsum[13-18]{wagahai}

\begin{figure*} % 目一杯拡げて2つ並べてみる
  \centering
  \myGraph{状態空間モデルによる各成分の推定}
  \myGraph{状態空間モデルによる平均の推定}
  \sidecaption{状態空間モデルの推定.
    \label{fig:2}}
\end{figure*}

\section{鼠以外の御馳走}
\label{sec:org35c2a12}
\jalipsum[19-24]{wagahai}
\begin{marginfigure}
  \centering
  \myGraph*{} % marginに合わせて表示
  \caption{$Q_{\mathrm{cycle}}$ の検討について.
    \label{fig:4}}
\end{marginfigure}

\section{放蕩家について}
\label{sec:org072125a}
\jalipsum[25-]{wagahai}

\begin{otherlanguage}{english}
  % babel系が若干悪さをするので英語にして回避
  \printbibliography[title=参考文献]
\end{otherlanguage}
\end{document}